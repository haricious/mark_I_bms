\documentclass[12pt,a4paper]{article}

\usepackage[utf8]{inputenc}
\usepackage{mathptmx}
\usepackage{amsmath}
\usepackage{amssymb}
\usepackage{geometry}
\usepackage{setspace}
\usepackage{graphicx}
\usepackage{float}
\usepackage{caption}
\usepackage{subcaption}
\usepackage{booktabs}
\usepackage{fancyhdr}
\usepackage[colorlinks=true, linkcolor=black, urlcolor=blue, citecolor=black]{hyperref}

\geometry{margin=1in}
\onehalfspacing
\setlength{\parskip}{0.5em}

\pagestyle{fancy}
\fancyhf{}
\rhead{\small \textit{Joint SoC \& SoH Estimation}}
\lhead{\small \textit{Internship Report}}
\cfoot{\thepage}

\begin{document}

\begin{titlepage}
\begin{center}

\vspace*{0.5cm}
{\Large \textbf{INTERNSHIP REPORT}}\\[1.5cm]

{\LARGE \textbf{Joint State-of-Charge and State-of-Health Estimation}}\\
\vspace{0.2cm}
{\LARGE \textbf{for Battery Management Systems}}\\
\vspace{0.2cm}
{\Large \textbf{Using MATLAB/Simulink}}\\[2cm]

\textit{Submitted by}\\[0.3cm]
{\Large \textbf{HARI SANKAR SARAVANAN}}\\
\textbf{Register Number: 921623106034}\\[0.2cm]
Bachelor of Engineering – Electronics and Communication Engineering\\
SBM College of Engineering and Technology, Dindigul\\[2cm]

\textbf{Internship Duration}\\
05 January 2026 -- 19 January 2026\\[2cm]

\textbf{Submitted to}\\[0.3cm]
\textbf{Centre for Rural Energy}\\
The Gandhigram Rural Institute (Deemed to be University)\\
Dindigul, Tamil Nadu\\[1.5cm]

\vfill
\today

\end{center}
\end{titlepage}

\newpage

\section*{Abstract}
This report describes the development of a combined State-of-Charge and State-of-Health estimation system using MATLAB and Simulink. An Extended Kalman Filter is applied on a 12V, 7Ah lead-acid battery model to improve tracking accuracy and compensate for modelling uncertainties.

\tableofcontents
\listoffigures
\newpage

\section{Internship Overview}

\begin{table}[H]
\centering
\caption{Internship Summary}
\renewcommand{\arraystretch}{1.5}
\begin{tabular}{ll}
\toprule
Parameter & Details \\
\midrule
Project Title & Joint SoC and SoH Estimation for BMS \\
Student Name & Hari Sankar Saravanan \\
Duration & 15 Working Days (Jan 05 -- Jan 19, 2026) \\
Domain & Battery Management System \\
Tools Utilized & MATLAB R2025b, Simulink, Simscape Electrical \\
Key Objectives & Battery Modeling, EKF Implementation, Parameter Estimation \\
\bottomrule
\end{tabular}
\end{table}

\section{System Architecture}

The system brings together the battery model, the monitoring logic, and the estimation algorithms.

\begin{figure}[H]
\centering
\includegraphics[width=1.0\textwidth]{docs/Main_sys.png}
\caption{Top-level system architecture}
\end{figure}

\section{Battery Pack Modeling}

\subsection{Pack Configuration}

\begin{figure}[H]
\centering
\includegraphics[width=0.85\textwidth]{docs/Building a Battery Pack/1_batpack.png}
\caption{Battery pack used for simulation}
\end{figure}

\subsection{Circuit Design}

\begin{figure}[H]
\centering
\includegraphics[width=0.85\textwidth]{docs/Building a Battery Pack/2_charging_discharging_ckt.png}
\caption{Charging and discharging circuit}
\end{figure}

\subsection{Operational Modes}

\begin{figure}[H]
\centering
\begin{subfigure}[b]{0.48\textwidth}
\centering
\includegraphics[width=\textwidth]{docs/Building a Battery Pack/3_charging.png}
\caption{Charging}
\end{subfigure}
\hfill
\begin{subfigure}[b]{0.48\textwidth}
\centering
\includegraphics[width=\textwidth]{docs/Building a Battery Pack/4_discharging.png}
\caption{Discharging}
\end{subfigure}
\caption{Current flow during operation}
\end{figure}

\subsection{Charging Algorithm}

\begin{figure}[H]
\centering
\includegraphics[width=0.8\textwidth]{docs/Building a Battery Pack/5_CC-CV Algorithm.png}
\caption{CC–CV charging method}
\end{figure}

\section{Battery Equivalent Circuit Model}

\begin{figure}[H]
\centering
\includegraphics[width=0.7\textwidth]{docs/batt_model.png}
\caption{First-order RC battery model}
\end{figure}

\begin{equation}
V_t = V_{ocv}(\text{SoC}) - I R_0 - V_{rc}
\end{equation}

\section{Simulation Parameters}

\begin{table}[H]
\centering
\caption{Battery and Circuit Parameters}
\renewcommand{\arraystretch}{1.2}
\begin{tabular}{lcc}
\toprule
Parameter & Symbol & Value \\
\midrule
Nominal Voltage & $V_{nom}$ & 12 V \\
Rated Capacity & $C_{rated}$ & 7 Ah \\
Nominal Capacity & $C_{nom}$ & 25200 C \\
Internal Resistance & $R_0$ & 0.05 $\Omega$ \\
Polarization Capacitance & $C_1$ & 1000 F \\
Load Resistance & $R_L$ & 2 $\Omega$ \\
Bleeder Resistance & $R_2$ & 10 k$\Omega$ \\
Temperature & $T$ & 285.15 K \\
\bottomrule
\end{tabular}
\end{table}

\begin{table}[H]
\centering
\caption{SoC Estimation and EKF Parameters}
\renewcommand{\arraystretch}{1.2}
\begin{tabular}{lcc}
\toprule
Parameter & Symbol & Value \\
\midrule
Sampling Time & $\Delta t$ & Variable Step \\
Initial Covariance & $P_0$ & $1\times10^{-3}$ \\
Process Noise & $Q$ & $1\times10^{-6}$ \\
Measurement Noise & $R$ & $1\times10^{-2}$ \\
Jacobian Perturbation & $\Delta$ & $1\times10^{-3}$ \\
Reference Temperature & $T_{ref}$ & 298.15 K \\
Temperature Coefficient & $\alpha_T$ & $-0.003$ \\
Minimum $R_0$ & -- & $0.2 R_{0,ref}$ \\
\bottomrule
\end{tabular}
\end{table}

\begin{table}[H]
\centering
\caption{OCV–SoC Lookup Table}
\renewcommand{\arraystretch}{1.1}
\begin{tabular}{cc}
\toprule
SoC & OCV (V) \\
\midrule
0.0 & 11.0 \\
0.1 & 11.5 \\
0.2 & 11.8 \\
0.3 & 12.0 \\
0.4 & 12.2 \\
0.5 & 12.4 \\
0.6 & 12.6 \\
0.7 & 12.8 \\
0.8 & 13.0 \\
0.9 & 13.5 \\
1.0 & 13.8 \\
\bottomrule
\end{tabular}
\end{table}

\section{OCV–SoC Characterization}

\subsection{Equivalent Circuit Model}

\begin{figure}[H]
\centering
\includegraphics[width=0.8\textwidth]{docs/OCV-SOC plot/EQ_model/ocv-soc_pb_eq_ckt_dig.png}
\caption{Digital OCV–SoC model}
\end{figure}

\begin{figure}[H]
\centering
\includegraphics[width=0.8\textwidth]{docs/OCV-SOC plot/EQ_model/ocv-soc_pb_eq_ckt_output.png}
\caption{OCV–SoC output characteristics}
\end{figure}

\subsection{Pre-built Model}

\begin{figure}[H]
\centering
\includegraphics[width=0.8\textwidth]{docs/OCV-SOC plot/Pre-built model/1_pre-built_ckt.png}
\caption{Pre-built OCV model}
\end{figure}

\begin{figure}[H]
\centering
\includegraphics[width=0.8\textwidth]{docs/OCV-SOC plot/Pre-built model/batt_measurements.png}
\caption{Measured OCV data}
\end{figure}

\section{Discharge Characteristics}

\begin{figure}[H]
\centering
\includegraphics[width=0.9\textwidth]{docs/Studying the Discharge I-V of a Lead-acid battery of 12V 7Ah/IV_12V_7Ah_output.png}
\caption{Discharge I–V curve}
\end{figure}

\section{Estimation Algorithm Blocks}

\subsection{SoC Estimation}

\begin{figure}[H]
\centering
\includegraphics[width=0.8\textwidth]{docs/cc_block.png}
\caption{Coulomb counting method}
\end{figure}

\subsection{Temperature Compensation}

\begin{figure}[H]
\centering
\includegraphics[width=0.8\textwidth]{docs/temp_block.png}
\caption{Temperature validation}
\end{figure}

\subsection{State of Health}

\begin{figure}[H]
\centering
\includegraphics[width=0.8\textwidth]{docs/SoH_block.png}
\caption{SoH estimation block}
\end{figure}

\section{Simulation Results}

\subsection{Dashboard Overview}

\begin{figure}[H]
\centering
\includegraphics[width=1.0\textwidth]{docs/final_out_1.png}
\caption{Dashboard for real-time estimation}
\end{figure}

\subsection{SoC Estimation Accuracy}

\begin{figure}[H]
\centering
\includegraphics[width=1.0\textwidth]{docs/final_out_2.png}
\caption{True SoC compared with EKF estimate}
\end{figure}

\subsection{SoH Tracking}

\begin{figure}[H]
\centering
\includegraphics[width=1.0\textwidth]{docs/final_out_3.png}
\caption{Resistance and capacity evolution}
\end{figure}

\section{Work Log}

\begin{table}[H]
\centering
\caption{Daily Activity Log}
\renewcommand{\arraystretch}{1.3}
\begin{tabular}{cl}
\toprule
Day & Task \\
\midrule
1--3 & Battery plant model development \\
4--6 & OCV characterization and Coulomb counting \\
7--9 & EKF formulation and tuning \\
10--12 & SoH logic development \\
13--15 & Final integration and testing \\
\bottomrule
\end{tabular}
\end{table}

\section{Conclusion}
The developed joint estimator provided accurate State-of-Charge prediction and consistent State-of-Health tracking. The Extended Kalman Filter improved stability and reduced drift, while the SoH logic captured both resistance change and capacity fade effectively.

\end{document}
